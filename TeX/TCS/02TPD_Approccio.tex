L'approccio utilizzato sarà di tipo black-box: verranno testate le funzionalità implementate specificate nel RAD senza considerare l'implementazione del codice. Il testing verrà eseguito in modo bottom-up, per evitare l'implementazione di stub ritenuti costosi a causa del paradigma \emph{object-oriented} utilizzato, partendo dalle Repositories che fungono da astrazione per gli oggetti di accesso ai dati permanenti, implementando degli Unit test che testano i singoli metodi in maniera procedurale, poiché tali classi non dipendono da alcuno stato. Proseguendo, verranno verificati i vari Services implementati, senza l'utilizzo di stub poiché sono strettamente correlate alle repositories utilizzate. Terminando, verranno testati gli oggetti Control e le interfacce grafiche (le \emph{view}) che essi ritornano, verificando che l'interfaccia tra la parte front-end e la back-end permetta effettivamente la comunicazione e la chiamata di procedure per effettuare azioni come la visualizzazione o il salvataggio di dati. Gli input per i vari test verranno scelti partizionando l'intero insieme di possibili input in classi di equivalenza, per ridurre il numero di input da utilizzare per la scelta dei vari test cases.