Esistono vari tipi di sistemi attualmente esistenti. In primis Play store, di Google, e App Store, di Apple, i quali sono i principali store che vengono preinstallati sui dispositivi attualmente in commercio. Questi store gestiscono la totalità delle applicazioni esistenti sul mercato attuale, indipendentemente dal tipo di applicazione, oltre che a rendere disponibili prodotti secondari come film o libri. Possono essere considerati come i punti di accesso primari al mondo software dei dispositivi mobile. Inoltre, esistono anche store secondari, con svariati obiettivi come quello di rendere disponibili solo applicazioni open-source e gratis, o fornire contenuti di tipo diverso ai propri utenti. Attualmente non esiste uno store dedicato ai videogiochi mobile, l’idea che più si avvicina è ristretta ai videogiochi per computer (Windows, Linux, MacOS). Lo store più famoso per questi videogiochi è Steam e rappresenta la validità di questa idea con il suo fatturato di 3.5 miliardi di dollari nel 2016.