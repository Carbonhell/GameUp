\section{Scopo del sistema}
Con il passare degli anni, il settore videoludico ha acquisito sempre più importanza, sia a livello di interesse della popolazione mondiale che al livello economico, con un mercato molto ricco sia dei prodotti effettivi sviluppati che degli strumenti atti allo sviluppo, in particolare per l’ecosistema mobile, ovvero dei dispositivi portatili come smartphone e tablet, oggigiorno fondamentali e posseduti dalla quasi totalità della popolazione dei paesi industrializzati. Attualmente il mercato è governato da due colossi: Google per i dispositivi Android e Apple per quelli iOS, in costante competizione per aggiudicarsi la fetta più ampia del mercato. Prendendo in considerazione il mercato videoludico di questi dispositivi, la situazione attuale lascia a desiderare, in primis per il modo in cui le applicazioni videoludiche vengono gestite all’interno degli “store” ufficiali di queste due piattaforme, rendendo la vita difficile sia agli utenti finali, per i quali risulta difficile trovare un videogioco adatto ai loro gusti, sia agli sviluppatori, obbligati a sottostare alle regole e ai costi di entrambe le piattaforme, ottenendo un servizio mediocre e incompleto per gestire tutti i vari aspetti del proprio videogioco. GameUp si prefissa come obiettivo quello di rendere disponibile un singolo store unificato per entrambe le piattaforme, specifico per questa ampia fetta di mercato del mondo videoludico, rendendo disponibili funzionalità utili e già esistenti in modo generico, ma non specifiche per l’ambito mobile, così da semplificare la ricerca di videogiochi da parte degli utenti finali e la gestione da parte degli sviluppatori.

\section{Ambito del sistema}
Il sistema GameUp vuole semplificare la vita in due modi diversi, per i due principali tipi di utenti finali: videogiocatori e sviluppatori di videogiochi. Per i primi, si vuole offrire un semplice e potente sistema di ricerca di contenuti basato sulle preferenze indicate dall’utente, come ad esempio la ricerca per genere (giochi di azione, avventura, gestionali…) o per determinate caratteristiche (multigiocatore o giocatore singolo, il tema generale del gioco…), oltre che a dare la possibilità di scrivere una recensione per un gioco posseduto e a partecipare alla comunità di quel gioco in modo simile ad un forum. Per gli sviluppatori, bisogna rendere semplice l’intera gestione del videogioco, dall’inserimento di esso agli aggiornamenti del gioco, come se fossero amministratori del sito ma solo relativi ai contenuti da loro gestiti, riguardanti il loro videogioco (inclusa la parte di forum e le recensioni).

\section{Obiettivi e criteri di successo del progetto}
Gli obiettivi principali del progetto sono:
\begin{itemize}
\item Rendere disponibile il catalogo dei giochi gestiti dal sistema, offrendo dei filtri per rendere più semplice la ricerca e permettendo la scelta di un gioco da parte dell’utente;
\item Sviluppare il pannello di gestione dei videogiochi per gli utenti sviluppatori;
\item Dare la possibilità agli utenti e gli sviluppatori di un videogioco di comunicare sulla stessa piattaforma dove viene effettuato il download, permettendo una partecipazione più semplice ed efficace;
\item Creare il pannello di gestione per il sito per gli amministratori, così da poter approvare nuovi videogiochi o gestire eventuali problemi;
\item Fornire l’intero sistema sia ai dispositivi Android che quelli iOS, oltre che su quante più piattaforme possibili, considerando la possibilità di nuovi concorrenti futuri;
\end{itemize}

\section{Definizioni, acronimi e abbreviazioni}
\begin{itemize}
\item GameUp: Nome della piattaforma da sviluppare;
\item Utente: Qualsiasi attore che accede alla piattaforma;
\item Sviluppatore: Un utente che può anche richiedere l’inserimento di un videogioco nella piattaforma e che ha accesso alla gestione dei propri videogiochi pubblicati;
\item Amministratore: Utente con permessi completi sul sistema GameUp;
\item Mobile: Dispositivi portatili, come smartphone o tablet;
\item Store: Catalogo di prodotti, simile ad un e-commerce;
\end{itemize}
