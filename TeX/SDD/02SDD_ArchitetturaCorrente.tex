Attualmente non esiste un sistema con lo stesso scopo di quello proposto, bensì il sistema proposto trae spunto dalle implementazioni di sistemi simili ma con obiettivi o target diversi. I principali sistemi da cui il sistema corrente prende spunto sono:
\begin{itemize}
	\item \href{https://store.steampowered.com/?l=italian}{Steam}: Uno dei primi e-commerce dedicati ai videogiochi, avente come target il mondo videoludico su computer. Esso offre dei servizi in forma web molto simili al sistema proposto, ma contiene videogiochi unicamente per computer, escludendo completamente altre piattaforme. Anche con un target ristretto, vanta un guadagno di oltre 4 miliardi di dollari annui.
	\item Android Play Store / iOS App Store: I punti di accesso principali sulla gran parte dei dispositivi mobile esistenti. Essi permettono il download di applicazioni compatibili per i dispositivi relativi (Dispositivi Apple per iOS App Store, Dispositivi Android di svariati produttori per Android Play Store), senza un determinato target. Da essi è possibile anche il download di videogiochi mobile, ma non è assolutamente lo scopo principale, non sono offerte funzionalità avanzate di ricerca dedicate a questo tipo di target e non esiste alcun tipo di comunità o forum all’interno di tali piattaforme.
\end{itemize}
Esistono, inoltre, servizi minori per piattaforme attualmente meno usate, come ad esempio Windows Phone, o anche la versione cinese dell’app store, dopo le recenti problematiche tra America e Cina. 