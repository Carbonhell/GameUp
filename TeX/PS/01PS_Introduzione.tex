Con la rivoluzione tecnologica attualmente in corso da vari decenni, in particolare negli ultimi due, abbiamo assistito ad un miglioramento esponenziale dell'hardware per lo svolgimento di operazioni complesse relative all'informatica, sia in termini di performance che di grandezza delle apparecchiature. Ciò ha consentito la produzione di computer tascabili, comunemente conosciuti come smartphone, attualmente potenti come computer di fascia alta del decennio scorso. Tra i vari settori che hanno approfittato di tali strumenti, uno in particolare ha iniziato ad espandersi sempre di più: quello relativo all'\emph{entertainment}.\\
Il mondo videoludico tratta un mercato di \$145.7 miliardi nel 2019, e da \$160 miliardi nel 2020, anche in seguito alla recente pandemia che ha fornito molto tempo libero da passare dentro la propria abitazione. Una parte di questo fatturato è dovuto al mondo videoludico mobile, il quale ha generato da solo \$75.4 miliardi nel 2020, con un +11\% all'inizio della pandemia, dovuto al fatto che in molti non hanno computer potenti dedicati al \emph{gaming}, ma spesso hanno l'ultima versione dell'iPhone, o dispositivi di fascia alta Android semplicemente per moda o per le fotocamere di alto livello che permettono di condividere facilmente esperienze sui vari social.

\section{Problema}
Nonostante tutto ciò, un videogiocatore mobile deve per forza trovare videogiochi nuovi tramite le due piattaforme principali, che non sono dedicate a questo mondo in continua evoluzione, bensì sono le piattaforme generiche offerte dai due tipi di sistema operativo mobile per il download di applicazioni autorizzate. \\
Il problema è proprio questo: la mancanza di una piattaforma che sfrutti questo mercato, aiutando sia i videogiocatori ma anche gli sviluppatori che scelgono come target il \emph{mobile gaming} per permettere una classificazione facilitata, una ricerca semplice ed efficace, ma anche la costruzione di comunità relative ai singoli videogiochi, basandosi sulla legge di Reed, ovvero che l'efficacia di una rete (in questo caso, l'insieme di utenti connessi da un videogioco in comune) cresce esponenzialmente avendo un interesse in comune, ma anche dalla prova pratica offerta da soluzioni simili ma per mercati alternativi, come quello del \emph{pc gaming}, con Steam come rappresentante principale, il quale offre proprio questo tipo di funzionalità.
La nostra soluzione sarà quindi quella di adottare le funzionalità offerte dalle piattaforma attualmente esistenti e applicarle in questo ambito.